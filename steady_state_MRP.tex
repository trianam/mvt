%This work is licensed under the Creative Commons
%Attribution-ShareAlike 4.0 International License. To view a copy of
%this license, visit http://creativecommons.org/licenses/by-sa/4.0/ or
%send a letter to Creative Commons, PO Box 1866, Mountain View, CA
%94042, USA.

\input{header}


\begin{document}

\title[\acs{MRP} steady-state]{\textbf{\acl{MRP} -
    steady-state analysis}}
\date[\today]{\flushright \today}
\subtitle{MVT exam}
\institute[Uni. Firenze]{
  \includegraphics[width=2cm]{img/logoUnifi.eps}\\
%  \includegraphics[width=2cm]{img/logoUnifi.png}\\
  Universit\`a degli Studi di Firenze
}

\author[Martina - Papini]{
  \begin{center}
    \begin{tabular}{ll}
      Stefano \textsc{Martina}&Tommaso \textsc{Papini}\\
      \href{mailto:stefano.martina@stud.unifi.it}{stefano.martina@stud.unifi.it}&
      \href{mailto:tommaso.papini1@stud.unifi.it}{tommaso.papini1@stud.unifi.it}
    \end{tabular}
  \end{center}
}

\titlegraphic{
  \vspace{-0.5cm}
  \tiny
  \href{http://creativecommons.org/licenses/by-sa/4.0/}{\includegraphics[width=1cm]{img/logoCC.png}}
  This work is licensed under a
  \href{http://creativecommons.org/licenses/by-sa/4.0/}{Creative
    Commons Attribution-ShareAlike 4.0 International License}.
}

\newacro{MRP}{Markov Regenerative Process}
\newacro{SMP}{Semi-Markov Process}
\newacro{CTMC}{Continuous Time Markov Process}
\newacro{DTMC}{Discrete Time Markov Process}
\newacro{PN}{Petri Net}

\acrodefplural{MRP}[MRP]{Markov Regenerative Processes}
\acrodefplural{SMP}[SMP]{Semi-Markov Processes}
\acrodefplural{CTMC}[CTMC]{Continuous Time Markov Processes}
\acrodefplural{DTMC}[DTMC]{Discrete Time Markov Processes}
\acrodefplural{PN}[PN]{Petri Nets}

\begin{frame}[plain]
  \titlepage
\end{frame}

\section{Intro \acfp{MRP}}
\begin{frame}
  \frametitle{\acfp{MRP}}
  \begin{block}{Definition}
    A \acf{MRP} is a stochastic process that sooner or later, with
    probability one, will
    reach a \alert{regenerative} state (will be regenerated).
  \end{block}
  \pause
  \begin{block}{Regenerative state}
    A state where the process loses its memory. 
  \end{block}
  \pause
  \begin{center}    
    \begin{tikzpicture}[node distance=5mm]
      \visible<5->{\draw[fill=c1] (1,0) ellipse (3cm and 1.4cm)  node[right=1.5cm] {\acs{MRP}};}
      \visible<4->{\draw[fill=c2] (0.5,0) ellipse (2cm and 1cm) node[right=.5cm] {\acs{SMP}};}
      \visible<3->{\draw[fill=c3] (0,0) ellipse (1cm and 0.6cm) node {\acs{CTMC}};}
    \end{tikzpicture}
  \end{center}
\end{frame}

\begin{frame}
  \frametitle{The steady-state problem}
  \begin{block}{Transient probabilities}
    The probability distribution that the process will be in a certain
    state, after given $t$ time.
  \end{block}
  \pause
  \begin{block}{Steady-state}
    For ergodic systems, it represents the probability distribution
    that the
    process will be in a certain state, as time goes to infinity.
  \end{block}
  \pause
  \begin{itemize}
  \item ORIS current state:
    \begin{itemize}
    \item Transient analysis for \acf{MRP}
    \item Steady-state analysis for everything else
    \end{itemize}
    \pause
  \item Until now! \rotatebox[origin=c]{270}{:D}
    \pause
  \item \alert{Warning:} we assume that the \acs{MRP} is ergodic, so we don't check it
  \end{itemize}
\end{frame}

\section{General algorithm}
\begin{frame}
  \frametitle{\acf{MRP} steady-state analysis theory}
  General idea:
  \begin{enumerate}
  \item<2-> Calculate the embedded \acs{DTMC} steady-state on the
    regenerative states
  \item<3-> Calculate the sojourn times in each marking, after each
    regenerative state
  \item<4-> Combine the two above in order to calculate the \acs{MRP} steady-state
  \end{enumerate}
  \begin{center}
    \begin{tikzpicture}[node distance=1.5cm and 5mm, text width=3cm, align=center]
      \uncover<4->{\node (combine) [rect, fill=c1] {\acs{MRP} steady-state};}
      \uncover<2->{\node (embedded) [rect, fill=c2, above left=of combine.north] {Embedded \acs{DTMC} steady-state};}
      \uncover<3->{\node (sojourn) [rect, fill=c3, above right=of combine.north] {Sojourn times};}
      \uncover<4->{\draw[arrow] (embedded) -- (combine);}
      \uncover<4->{\draw[arrow] (sojourn) -- (combine);}
    \end{tikzpicture}
  \end{center}
\end{frame}

\subsection{Steady-state of the embedded \acs{DTMC} on regenerative
  states}
\begin{frame}
  \frametitle{\insertsubsection}
  \begin{block}{Steady-state in \acf{DTMC}}
    If the steady-state of a \acf{DTMC} exists and is unique (if is
    ergodic), then it's calculated by solving for $v$ the linear
    system:
    \begin{equation*}
      \begin{cases}
        v=vP\\
        |v| = 1
      \end{cases}
    \end{equation*}
  \end{block}
  \pause
  \begin{itemize}
  \item We want to calculate the steady-state of the embedded \acs{DTMC} of the \acs{MRP} in
    the regenerative states
    \pause
  \item But we don't have $P$! \pause \rotatebox[origin=c]{270}{:'(}
  \end{itemize}
\end{frame}

\begin{frame}
  \begin{block}{Reaching probability feature}
    \begin{itemize}
    \item We add a new \alert{reaching probability feature} to each
      state: \jj{class ReachingProbabilityFeature}
      % \pause
    \item Inside \jj{SteadyStateInitialStateBuilder}: set it to
      1
      %\pause
    \item Inside \jj{SteadyStatePostProcessor}: multiply the
      parent's reaching probability by the probability to chose a
      certain child
    \end{itemize}
  \end{block}
  %\pause
  If we run a transient analysis on the \acf{PN} we get:
  \begin{itemize}
  \item \jj{regenerationClasses}
  \item {\scriptsize\jj{Map<DeterministicEnablingState,Map<DeterministicEnablingState,Set<State>>>}}
  \item sum reaching probability feature of each \jj{State} to compute
    elements of \alert{$P$}
  \end{itemize}
\end{frame}

\begin{frame}
  Now we can solve the linear system
  \begin{equation*}
    \begin{cases}
      v=vP\\
      |v| = 1
    \end{cases}
    =
    \begin{cases}
      (P'-I)v'=0\\
      \sum_i v_i = 1
    \end{cases}
  \end{equation*}
  \begin{itemize}
  \item \jj{RealMatrix} \& \jj{RealVector}
  \item QR decomposition solver
    \begin{itemize}\tiny
    \item \jj{DecompositionSolver solver = new QRDecomposition(coefficients).getSolver();}
    \item \jj{RealVector steadyState = solver.solve(constants);}
    \end{itemize}
  \item Convert \jj{steadyState} into a \jj{Map<DeterministicEnablingState,BigDecimal>}
  \end{itemize}
\end{frame}

\subsection{Sojourn time $a_{ij}$}
\begin{frame}
  \frametitle{\insertsubsection}
  \begin{block}{Definition}
    The sojourn time \alert{$a_{ij}$} represents the average time spent in the
    \alert{$j$-th marking} after the (last) \alert{$i$-th regeneration}.
  \end{block}
\end{frame}

\begin{frame}
  \frametitle{How to compute $a_{ij}$?}
  $a_{ij}$ is:
  \begin{itemize}
  \item sum of avg time spent in marking $j$ occurrences
    \begin{itemize}
    \item avg time before each variable fires
      \begin{itemize}
      \item condition each variable to be the minimum (i.e. the one
        that fires)
      \item compute avg time before that variable fires (thanks Marco!)
      \end{itemize}
    \end{itemize}
  \end{itemize}
  \begin{center}
    \begin{tikzpicture}[myMindmap]
      \begin{scope}[concept color=mmc2]
        \node(l) {i}
        child[concept color=mmcb] { node {}
          child { node{} }
          child[concept color=mmc1] { node{j}
            child[concept color=mmcb] { node{} }
            child[concept color=mmcb] { node{} }
          }
        }
        child[concept color=mmcb] { node{} }
        child[concept color=mmcb] { node{} 
          child[concept color=mmc1] { node{j}
            child[concept color=mmcb] { node {} }
          }
        };        
      \end{scope}
      \begin{scope}[concept color=mmc1]
        \node[right=5cm of l] {j}
        child[concept color=mmcb] { node[label={[label distance=2.5mm]50:$v_1$}] {} }
        child[concept color=mmcb] { node[label={[label distance=0.1mm]93:$v_2$}] {} }
        child[concept color=mmcb] { node[label={[label distance=2.5mm]135:$v_3$}] {} };
      \end{scope}
    \end{tikzpicture}
  \end{center}
\end{frame}

\begin{frame}
  \frametitle{When to compute $a_{ij}$?}
  During the transient analysis!
  \begin{itemize}
  \item transient analysis generates succession trees for each
    regenerative state
    \begin{itemize}
    \item regenerative state as root
    \item following regenerative states as leaves
    \item reachable markings as inner nodes
    \end{itemize}
  \item during the tree generation compute and accumulate $a_{ij}$ for
    each marking occurrence found
  \end{itemize}
\end{frame}
\subsection{\acf{MRP} steady-state}
\begin{frame}
  \frametitle{\insertsubsection}
  Let's combine the embedded \acs{DTMC} steady-state and the sojourn
  times!
  \pause
  \begin{equation*}
    \pi_j = \frac{\sum_i v_i a_{ij}}{K}
  \end{equation*}
  \pause
  \begin{itemize}
  \item We multiply the sojourn time in the marking $j$ after the
    regeneration $i$ by the probability of reaching the $i$-th regeneration
    \pause
  \item We do this for each regeneration that leads to the marking $j$
    before another regeneration
    \pause
  \item $K$ is a normalization factor calculated as the sum of $\pi_j$
  \end{itemize}
\end{frame}
\section{Test}
\begin{frame}
  \frametitle{\insertsection}
  
\end{frame}
\end{document}